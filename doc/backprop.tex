\documentclass[12pt,a4paper]{article}

\usepackage[utf8]{inputenc}
\usepackage[portuguese]{babel}
\usepackage{booktabs}
\usepackage{listings}
\usepackage{xcolor}
 
\definecolor{codegreen}{rgb}{0,0.6,0}
\definecolor{codegray}{rgb}{0.5,0.5,0.5}
\definecolor{codepurple}{rgb}{0.58,0,0.82}
\definecolor{backcolour}{rgb}{0.95,0.95,0.92}
 
\lstdefinestyle{mystyle}{
    backgroundcolor=\color{backcolour},   
    commentstyle=\color{codegreen},
    keywordstyle=\color{magenta},
    numberstyle=\tiny\color{codegray},
    stringstyle=\color{codepurple},
    basicstyle=\ttfamily\footnotesize,
    breakatwhitespace=false,         
    breaklines=true,                 
    captionpos=b,                    
    keepspaces=true,                 
    numbers=left,                    
    numbersep=5pt,                  
    showspaces=false,                
    showstringspaces=false,
    showtabs=false,                  
    tabsize=2
}
 
\lstset{style=mystyle}

\begin{document}

\begin{titlepage}
	\centering
	{\scshape\LARGE Universidade Federal do Pará \par}
	\vspace{1cm}
	{\scshape\Large Redes Neurais Artificiais\par}
	\vspace{1.5cm}
	{\huge\bfseries Implementação e Aplicação de Algoritmo de Backpropagation\par}
	\vspace{2cm}
	{\Large Helder Mateus dos Reis Matos\par}
	\vfill
	Dra. Adriana Rosa Garcez Castro
	\vfill
	{\large \today\par}
\end{titlepage}

\tableofcontents

\section{Introdução}

\section{Estrutura de uma Rede Neural Artificial}

\section{Backpropagation}

\section{Implementação}
A rede foi organizada em uma lista de três camadas, uma de entrada, uma escondida de e uma de saída, e cada camada é estruturada como um dicionário que, inicialmente, guarda os pesos sinápticos dos neurônios. Durante as fases de alimentação adiante e retropropagação, esse léxico recebe valores de saída da rede e de ajuste de pesos (regra delta).

\subsection{Inicializando pesos}
Os pesos são valores aleatórios uniformemente distribuídos, entre 0 e 1. O número de entradas de cada neurônio da camada escondida é equivalente à quantidade de neurônios na camada de entrada mais um bias, assim como o número de entradas de cada neurônio da camada de saída é equivalente à quantidade de neurônios na camada de entrada mais um bias.
\lstinputlisting[language=Python, firstline=4, lastline=17]{backpropagation.py}

\subsection{Feedforward (Alimentação Adiante)}

\subsection{Backpropagation (Retropropagação)}

\subsection{Treino}

\subsection{Validação}

\section{Aplicação}
\subsection{Escolha e tratamento do Dataset}
\subsection{Aplicação na Rede Neural}
\subsection{Análise dos Resultados}
\subsubsection{Erro Médio Quadrático}
\subsubsection{Saída Desejada $\times$ Saída Obtida}
\subsubsection{Comparações para Diferentes Topologias}
\section{Conclusão}
\section{Referências}
\section{Anexos}

\end{document}